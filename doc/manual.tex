% Time-stamp: <06/03/24 18:52:45 wash>

\documentclass[11pt]{article}
\usepackage{graphicx}
\usepackage{hyperref}
\usepackage{verbatim}
\usepackage{rotating}
\usepackage{tabularx}
\usepackage{booktabs}
\usepackage{afterpage}
\usepackage{times}
\usepackage{url}
\usepackage{caption2}
\usepackage[sf,bf]{titlesec}
\usepackage{fancyhdr} 
\usepackage[a4paper,includehead=true,bottom=3.5cm,top=3cm,left=4cm,right=3cm]{geometry} 


% header and footer
\pagestyle{fancy} 
\renewcommand{\sectionmark}[1]{\markboth{\thesection.\ #1}{}}
\fancyhead{}
\lhead{\sf \nouppercase{\leftmark}}
\rhead{\sf \thepage}
\fancyfoot{}

% paragraph
\setlength\parindent{0pt} 
\parskip2ex

% figure and table captions
\makeatletter
\renewcommand{\fnum@figure}[1]{Fig.~\thefigure.}
\renewcommand{\fnum@table}[1]{Tab.~\thetable. }
\makeatother
\renewcommand\captionfont{\sffamily \small}
\renewcommand\captionlabelfont{\bfseries \small}
\setcaptionmargin{0.05\textwidth}
\makeatletter \renewcommand\@biblabel[1]{#1.} \makeatother


%\renewcommand{\baselinestretch}{1.5}

\makeatletter
\makeatother

\pagenumbering{arabic}

\begin{document}
\title{Prediction of structural noncoding RNAs with RNAz} 
\author{Stefan Washietl\footnote{Department of Theoretical Chemistry, University of
  Vienna, W{\"a}hringerstra{\ss}e 17, A-1090 Wien, Austria}} 
%\maketitle

\thispagestyle{empty}

\vspace*{5cm}

\huge 
\sf

RNAz 1.0

\vspace*{-0.8cm}
\rule{\textwidth}{1mm}

\large
\vspace*{-0.3cm}
\hfill Predicting structural noncoding RNAs

\vspace{3cm}

\vfill

%\begin{center}
Stefan Washietl\\Department for Theoretical
Chemistry\\University Vienna

wash@tbi.univie.ac.at\\
http://www.tbi.univie.ac.at/\raisebox{-0.9ex}{\~{ }}wash/RNAz

%\end{center}
\normalfont

\newpage
\thispagestyle{empty}
\vspace*{0.9\textheight}

\begin{center}
  \$Id: manual.tex,v 1.2 2006/03/24 17:56:08 wash Exp $\ $\$
\end{center}


\newpage

\pagenumbering{roman}

\tableofcontents

\newpage


\section*{Preface}

The function of many noncoding RNAs (ncRNAs) depend on a defined secondary
structure. RNAz detects evolutionary conserved and thermodynamically stable
RNA secondary structures in multiple sequence alignments and thus
efficiently filters for candidate ncRNAs.

There are two main goals of this document. First we want to give you
detailed technical advice on how to use RNAz. Second, we want to help you
to get a well-founded understanding of the results you get from RNAz. We
want to assist in a sensible interpretation of RNAz predictions ---
leading, as we hope, to reasonable conclusions for your application.

This document is largely based on a draft for a book chapter and it is thus
organized in an idiosyncratic way. Until there is more time to write a
dedicated tutorial and manual we will keep this organization.

We start with a short introduction to the problem of \emph{de novo}
prediction of ncRNAs and the RNAz algorithm. In the next part, we explain
how to install RNAz and all necessary helper programs on your system. Next,
we demonstrate the basic usage of RNAz including the correct formatting of
the input alignments. More advanced techniques which require pre-processing
steps of the input alignments are discussed afterwards. In the last
section, we show you how to conduct a RNAz screen of a large number of
automatically generated alignments on the example of genome-wide screen of
\emph{Saccharomyces cerevisiae}.


\newpage

\pagenumbering{arabic}

\section{Introduction}

\subsection{Prediction of noncoding RNAs}

In contrast to protein-gene finders which are routinely used for genome
annotation, noncoding RNA (ncRNA) gene finders are still in their infancy.
The main reason that hinders systematic \emph{de novo} prediction of ncRNAs
is that there are no common statistically significant features in primary
sequence (e.g. open reading frames or codon bias) which could be exploited
for efficient algorithms.

It is even not clear what we define as ``ncRNA''. There is no doubt that
independent ``RNA genes'' with a defined molecular function such as tRNAs,
microRNAs, or snoRNAs should be called ncRNAs. But the situation is not
always that clear. The transcriptional activity of at least mammalian
genomes is much more complex than anticipated \cite{frith05}. We see
mRNA-like ncRNAs, non-polyadenylated RNAs from both intronic and intergenic
regions, overlapping transcripts, extensive antisense transcription, and
transcribed protein-pseudogenes. In addition, there is a recent example of
a noncoding transcript that only is expressed to interfere with and
downregulate the transcription of a neighboring gene but the produced RNA
molecule itself does not have any obvious function \cite{martens04}. There
is even an example of a functional RNA encoding a protein
\cite{chooniedass-ko04}. The spectrum of ncRNAs and their mode of action is
very heterogeneous. One can safely assume that the full spectrum of
functions is not yet discovered and that a general ncRNA gene finder is an
unrealistic goal even in the long term.

However, there is a subclass of ncRNAs which --- with the help of
comparative genomics --- can be predicted with fair accuracy.
\emph{Structural} ncRNAs have a defined and evolutionary conserved
secondary structure which is of functional importance. Most of the well
known ``classical'' ncRNAs, as for example tRNA, rRNA, RNAse P, or SRP RNA,
are of this class. Pioneering work in the prediction of structural ncRNAs
by comparative genomics was done by Rivas \& Eddy. QRNA predicts conserved
RNA secondary structures on pairwise alignments using a probabilistic
approach based on a stochastic context free grammar to model RNA structure
\cite{rivas01a,rivas01,mccutcheon03}. RNAz \cite{washietl05a} takes a
different approach. It is based on minimum free energy (MFE) structure
prediction algorithms \cite{zuker81,hofacker94a}. It relies on the fact
that structural RNAs have two characteristic features: (i) unusual
thermodynamic stability and (ii) conservation of secondary structure. The
following section outlines the basic principles of RNAz.

\subsection{The RNAz approach}

\subsubsection{Thermodynamic stability}
\label{sec:therm-stab}

It is easy to calculate the MFE as a measure of thermodynamic stability for
a sequence using e.g. RNAfold \cite{hofacker94a}. However, the MFE depends
on the length and the base composition of the sequence and is, therefore,
difficult to interpret in absolute terms. RNAz calculates a normalized
measure of thermodynamic stability by comparing the MFE $m$ of a given
(native) sequence to the MFEs of a large number of random sequences of the
same length and base composition. A $z$-score is calculated as
$z=(m-\mu)/\sigma$, where $\mu$ and $\sigma$ are the mean and standard
deviations, resp., of the MFEs of the random samples. Negative $z$-scores
indicate that a sequence is more stable than expected by chance.  RNAz does
not actually sample random sequences but approximates $z$-scores, which is
much faster but of the same accuracy.

\subsubsection{Structural conservation}
\label{sec:struct-cons}

RNAz predicts a consensus secondary structure for an alignment by using the
RNAalifold approach \cite{hofacker02}. RNAalifold works almost exactly as
single sequence folding algorithms (e.g. RNAfold), with the main difference
that the energy model is augmented by covariance information. Compensatory
mutations (e.g. a CG pair mutates to a UA pair) and consistent mutations
(e.g. AU mutates to GU) give a ``bonus'' energy while inconsistent
mutations (e.g. CG mutates to CA) yield a penalty. This results in a
\emph{consensus} MFE $E_A$. RNAz compares this consensus MFE to the
\emph{average} MFE of the individual sequences $\bar E$ and calculates a
structure conservation index: $\textrm{SCI}= E_A/\bar E $. The SCI will be
high if the sequences fold together equally well as if folded individually.
On the other hand, SCI will be low if no consensus fold can be found.

\subsubsection{Putting it together}

The two independent diagnostic features of structural ncRNAs, $z$-score and
SCI, are finally used to classify an alignment as ``structural RNA'' or
``other''. For this purpose, RNAz uses a support vector machine (SVM)
learning algorithm which is trained an a large test set of well known
ncRNAs.

Using RNAz, it is possible to efficiently screen alignments for functional
RNA secondary structures. It is important to note that RNAz cannot
distinguish functional RNA elements which are part of ncRNAs from elements
which are \emph{cis}-regulatory elements of mRNAs.

\subsection{General remarks and typographical conventions}

There is no graphical user interface for RNAz. All steps are carried out on
a command-line (terminal). Lines starting with a ``\#'' are commands and
you should type them into your terminal window, followed by pressing
return. The ``\#'' sign stands for your command line prompt and may look
different on your system. If a command is too long for one line in this
book it is separated by a backslash ``\texttt{$\backslash$}'' and continues
on the next line. Do \emph{not} input the backslash, simply type in the
command on one line.

All programs are implemented as filters, i.e. they read from the standard
input and write to the standard output. Therefore, we make use of the pipe
(``\texttt{|}'') and redirection operators (``\texttt{<}'',''\texttt{>}'').

You can get a online documentation on the usage of each program by using
the \texttt{--help} option, e.g.:

\begin{verbatim}
# RNAz --help
\end{verbatim}

For the Perl programs you get more detailed manual pages by using the
\texttt{--man} option. All manual pages are reproduced in Appendix
\ref{sec:manual-pages} in this manual.

Most command line options have a long (e.g. \texttt{--help}) and a short
(e.g. \texttt{-h}) form. For didactic reasons, we use long option names
throughout this manual.

\section{Materials}

\subsection{Hardware}

RNAz is generally fast. Small to medium sized data sets, as for example the
yeast screen in section~\ref{sec:large-scale-genomic}, can be analyzed
within reasonable time on a single modern desktop or even laptop computer.

\subsection{Operating system}

If available, we recommend to use a Linux/UNIX system for your analysis.
Also Mac OS X, in principle a full featured UNIX system, is an adaequate
platform.

Alternatively, you can run RNAz also on Microsoft Windows.  Most of the
methods described in this manual can be carried out on Windows without any
modification.


\subsection{Perl}
\label{sec:perl}

The RNAz program is bundled with a variety of helper programs
which are written in the Perl programming language. To run these programs
you need to have installed Perl on your system, which is most likely the
case on all Linux/UNIX systems and on Mac OS X.
 
Perl is not part of a standard Windows system. Windows users can download it
from \url{www.activestate.com}. Choose the latest ActivePerl MSI installer
package for Windows and simply follow the installation instructions. Make
sure that you have selected the ``Add Perl to the PATH environment
variable'' and ``Create Perl file extension association'' options during
installation.

\subsection{RNAz}

The RNAz program can be downloaded from:
\url{www.tbi.univie.ac.at/~wash/RNAz}. For the examples in this manual,
RNAz version 1.0 was used. For Linux/UNIX and OS X, download the file
\texttt{RNAz-1.0.tar.gz}. Windows user download the file
\texttt{RNAz-1.0-win32.msi}.

\subsection{Optional software}
\label{sec:optional-software}

Some advanced analysis steps (sections \ref{sec:website} and
\ref{sec:blast}) require additional software to be installed on your
system.

To create HTML formatted output of the results as described in section
\ref{sec:website} you will need to have installed the Vienna RNA package
(\url{www.tbi.univie.ac.at/RNA}) and the postscript interpreter Ghostscript
(\url{http://www.cs.wisc.edu/~ghost/}).

To perform automatic database searches of predicted ncRNA candidates you
need NCBI Blast (\url{ftp://ftp.ncbi.nih.gov/blast}).


\subsection{Example files}

Most of the example files used in this manual are part of the RNAz
package. If you want to reproduce the \emph{S. cerevisiae} screen described
in section~\ref{sec:large-scale-genomic} you can download the data file
from: \url{www.tbi.univie.ac.at/papers/SUPPLEMENTS/MiMB/}.

\section{Methods}

\subsection{Installation of RNAz}

\subsubsection{Linux/UNIX and OS X}

In the simplest case you can run the following series of commands to build
and install RNAz:

\begin{verbatim}
# tar -xzf RNAz-1.0.tar.gz
# cd RNAz-1.0 
# ./configure
# make
# su
# make install
\end{verbatim}

This requires root privileges and installs all files under the
\texttt{/usr/local} tree. The \texttt{RNAz} executable is installed in
\texttt{/usr/local/bin} and you should now be able to run the program (try
\texttt{RNAz --version} on a terminal window). If you do not have root
privileges or experience other problems (e.g. \texttt{gcc} compiler not
found) see note~\ref{sec:cust-inst-rnaz}.

The Perl programs are installed to \texttt{/usr/local/share/RNAz/perl}. To
make these programs available from other locations you can either add this
directory to your \texttt{PATH} of executables environment variable or copy
the Perl programs to an existing directory already in your \texttt{PATH}.
In case you are not familiar on how to run Perl programs refer to
note~\ref{sec:inst-runn-perl}.

\subsubsection{Microsoft Windows}
\label{sec:microsoft-windows}

To install RNAz on Windows simply double click on the
\texttt{RNAz-1.0-win32.msi} and follow the instructions. Open a console
window and type \texttt{RNAz --version} to test your installation.

\subsection{Installation of optional Software}
\label{sec:inst-opti-softw}

We cannot cover in detail the installation procedure of the optional
software. We just give an outline how to install the Vienna RNA package and
NCBI blast on a standard Linux system. Together with an existing
Ghostscript installation, this will allow you to run the examples in
sections \ref{sec:website} and \ref{sec:blast}. Windows and OS X users see
Note~\ref{sec:opti-softw-wind}.

To install the Vienna RNA package, get the latest
\texttt{ViennaRNA-X.X.tar.gz} file from \url{www.tbi.univie.ac.at/RNA}. The
package can be installed in exactly the same way as RNAz, using
\texttt{./configure} and \texttt{make}. Please refer to the
\texttt{INSTALL} document for detailed installation options. Make sure that
the Perl programs in the \texttt{Utils} directory are in your \texttt{PATH}
of executables.

To install NCBI Blast download the \texttt{blast-2.*.tar.gz}-package
matching your platform from
\url{ftp://ftp.ncbi.nih.gov/blast/executables/LATEST/}. Copy it to an
installation directory of your choice and ``untar'' it. The executables are
located in the \texttt{bin} subdirectory which you should add to your
\texttt{PATH} variable.

\subsection{Installation of example files}

Move the example file \texttt{yeast-examples.tar.gz} to a directory of your
choice and ``untar'' the file:

\begin{verbatim}
# tar -xzf yeast-examples.tar.gz
\end{verbatim}

If you are using Windows, download the file \texttt{yeast-examples.zip} and
unzip it in a directory of your choice.

\subsection{Basic usage of RNAz}

\subsubsection{Input alignment}

RNAz takes a multiple sequence alignment as input. RNAz does \emph{not}
align sequences, so you have to use other programs for creating your
alignments. If you prepare your alignments manually (in contrast to
automatic genome-wide alignments as in
section~\ref{sec:large-scale-genomic}) we recommend using Clustal W
\cite{thompson94}. It is an easy-to-use and widely available tool which
performs well on structural RNAs \cite{gardner05}. For hints on preparing
the alignments see note~\ref{sec:creat-input-alignm}.

RNAz can read two different alignment formats: Clustal W
(Fig.~\ref{fig:formats}A) and MAF (Fig.~\ref{fig:formats}B). The Clustal W
format is a concise format which is supported by many programs and thus
suitable for every-day use.

For genomic screens, however, it is necessary to exactly store the genomic
locations of aligned sequences. For this purpose, the MAF format was
developed which requires six fields for each sequence entry:

\begin{enumerate}

\item a unique identifier of the source sequence,
\item the start position of the aligned subsequence with respect to this
  source sequence,
\item the length of the aligned subsequence without gaps,
\item ``+'' or ``-'' indicating if the sequence is in the same reading
  direction of the source sequence or the reverse complement,
\item the sequence length of the complete source sequence,
\item the aligned subsequence with gaps.

\end{enumerate}


\begin{figure}
\centerline{\includegraphics*[width=11cm]{figs/formats.eps}}
\caption{Supported alignment formats and RNAz output. (A) Clustal W format,
  (B) MAF format (sequences have been shortened due to space restrictions),
  (C) Output of RNAz on the MAF file shown in (B).}
\label{fig:formats}
\end{figure}



The full specification of the format can be found here:
\url{http://genome.ucsc.edu/goldenPath/help/maf.html}. It should be noted
that RNAz and all other helper programs do not make use of field 5 and also
ignore the value of the ``\texttt{score=}'' field in the header line. So it
is possible to simply fill these fields with 0 or any other arbitrary
values, if the real values are not easily available.

The RNAz package contains several example files which are by
default installed to \texttt{/usr/local/share/RNAz/examples}. To run the
following examples change into this directory.

\subsubsection{Running RNAz}

As soon you have prepared your alignment you can immediately score it with
RNAz. In the simplest case you type:

\begin{verbatim}
# RNAz tRNA.maf
\end{verbatim}

The file \texttt{tRNA.maf} is that one shown in Fig.~\ref{fig:formats}B and
the command gives you the output shown in Fig.~\ref{fig:formats}C.

\subsubsection{Understanding the output}
\label{sec:understanding-output}

As described in the introduction, RNAz calculates various folding
characteristics to classify the alignment. These are displayed in the
header section of the RNAz output.

The mean single MFE is compared to the consensus MFE which results in the
SCI, a measure for structural conservation (section~\ref{sec:struct-cons}).
In this ideal example of a tRNA, we observe a very high SCI of 0.97. The
SCI depends on the mean pairwise identity and the number of sequences in
the alignment.  So, it is not possible interpret the significance of a
SCI-value in absolute terms.  As a rule of thumb, a SCI near or even above
the mean pairwise identity is ``good'' and might indicate structural
conservation.  For example, given an alignment with five sequences and a
mean pairwise identity of 60\%, a SCI of 0.75 can be regarded as strong
hint for a conserved fold. On the other hand, on a pairwise alignment with
90\% identity, SCI=0.75 does not indicate a conserved fold at all.

The second characteristic is thermodynamic stability, which is expressed as
the mean $z$-score of the sequences in the alignment (see section
\ref{sec:therm-stab}). $z$-scores of MFEs are not exactly normal
distributed, so you cannot directly give a statistical significance for
your $z$-score. However, mean $z$-scores below $-3$ or $-4$ generally
indicate very stable structures, that should arise only in rare cases by
chance. Also here, one has to consider the overall sequence divergence in
the alignment. On a pairwise alignment with 90\% identity a $z$-score of
$-4$ is much more likely to occur by chance than on an alignment of six
sequences with only 60\% identity.

Apart from SCI and $z$-score, there are a few other values displayed in the
RNAz output. If you are wondering what they mean, see note
~\ref{sec:addit-outp-valu}.

RNAz assists you in the final classification by providing an overall
``RNA-class probability'', or ``$P$-value''. It is important to know that
this is \emph{not} a $P$-value in a strict statistical sense, simply
because there is no underlying statistical model. Instead, RNAz uses a
rather \emph{ad hoc} machine learning technique to calculate this value.
If $P>$0.5, the alignment is classified as ``RNA''.  The false positive
rate at this cutoff was found to be $\approx$ 4\%, i.e. we expect 4
positive hits in 100 random alignments. For many applications it is useful
to set a more stringent cutoff of $P$=0.9 with an associated false positive
rate of $\approx$ 1\%. Reasons why estimations of false positives must
always be taken with caution are given in Note~\ref{sec:estim-false-posit}.

It turned out to be a useful practice to use $P=$0.5 and $P=$0.9 as two
main levels of significance. A more sophisticated interpretation of the
$P$-value without considering the other values is generally not useful. In
most cases you cannot say that, for example, a hit with $P=$0.97 is more
reliable than a hit with $P=$0.95. See Note \ref{sec:filt-sign-hits} on how
to assess the reliability of a hit based on other criteria.

In the lower part of the RNAz output you explicitely see the predicted
structures for your sequences. You get structure predictions for each
single sequence and a consensus structure prediction for the whole
alignment. The predicted structures are given below the sequences in a
``dot-bracket'' notation. Each base-pair in the secondary structure is
indicated by a pair of brackets: ``\texttt{(}'' and ``\texttt{)}''.
Unpaired bases are shown as dot: ``\texttt{.}''. Next to the structure you
see the MFE in kcal/Mol. You can get a graphical output by using RNAalifold
of the ViennaRNA package.

\subsection{Advanced usage of RNAz}

\subsubsection{Analyzing forward and reverse strand}

For a given alignment, a putative RNA can either be read in the forward
direction or in the reverse complementary direction. Therefore, both
reading directions should be scanned. By default, only the forward
direction is scored, but you can use the \texttt{--forward},
\texttt{--reverse} and \texttt{--both-strands} flags to explicitely specify the
reading direction. 

If you have a strong RNA signal in one strand you can observe in many cases
also a signal in the reverse complement.  Usually the signals (SCI,
$z$-score, consensus MFE) are stronger in the ``correct'' direction. In
most cases this also goes along with a better $P$ value.  That is not
always the case and, therefore, RNAz uses a separate SVM decision model to
predict the correct strand. Please note, that in version 1.0 this is still
an experimental feature. With the following command you can analyze both
strands of the tRNA and, in addition, activate the strand prediction:

\begin{verbatim}
# RNAz --both-strands --predict-strand tRNA.maf
\end{verbatim}

In this example, the signal from both strands are almost indistinguishable
and also the $P$-values are almost the same (0.993 and 0.999).
RNAz still suggests the correct (forward) strand and displays a
``strand class probability'':

\begin{verbatim}
# Strand winner: forward (0.88)
\end{verbatim}

\subsubsection{Scoring alignments with more than six sequences}
\label{sec:scor-large-alignm}

RNAz is currently limited to alignments with not more than six sequences.
If you have more than six sequences in your alignment, you have to reduce
the number either manually or use the \texttt{rnazSelectSeqs.pl} program to
filter your alignment before you put it into RNAz:

\begin{verbatim}
# rnazSelectSeqs.pl miRNA.maf | RNAz
\end{verbatim}

The file \texttt{miRNA.maf} contains 12 aligned microRNAs. With default
parameters,\\ \texttt{rnazSelectSeqs.pl} selects a subset of six sequences
trying to reach an optimal mean pairwise identity around 80\%.

The default behaviour can be customized in various ways (use
\texttt{--help} for details). The following command, for example, samples
three different alignments with four sequences each.

\begin{verbatim}
# rnazSelectSeqs.pl --num-seqs=4 --num-samples=3 miRNA.maf | RNAz
\end{verbatim}

By default, the first sequence in the alignment is always in the set of
selected sequences. This is the desired behaviour for genomic screens,
where one usually likes to retain a reference sequence.


\subsubsection{Scoring long alignments}
\label{sec:scor-long-alignm}

RNAz cannot score alignments longer than 400 columns. In practice, it is
generally advisable that you score long alignments, say $>$200 columns, in
shorter, overlapping windows. For general purpose screens we recommend a
window size of 120. This window size appears large enough to detect local
secondary structures within long ncRNAs and, on the other hand, small
enough to find short secondary structures without loosing the signal in a
much too long window.

The file \texttt{unknown.aln} contains a noncoding region conserved in
vertebrates. You can scan it for RNA secondary structures by typing:

\begin{verbatim}
# rnazWindow.pl --window=120 --slide=40 unknown.aln \
                | RNAz --both 
\end{verbatim}

If you look trough the results you see that RNAz does not predict an RNA in
this region. On UNIX like system you can add ``\texttt{| grep Prediction}''
to get a quick overview on the results. The \texttt{rnazWindow.pl} program
has numerous additional functions and will be used again in section
\ref{sec:large-scale-genomic}.

\subsection{Large scale genomic screens}
\label{sec:large-scale-genomic}

\subsubsection{Overview}

An analysis pipeline suitable for scanning a large number of genomic
alignments is outlined in Fig.~\ref{fig:pipeline}. In the following, we
demonstrate the usage of this pipeline on the example of a genomic screen
of \emph{Saccharomyces cerevisiae}. We want to describe the method as
general as possible and we will focus here mainly on technical details. A
paper describing the results of a comprehensive RNAz screen in yeast is in
preparation \cite{steigele06}.

\begin{figure}
  \centerline{\includegraphics*[width=0.8\textwidth]{figs/flowchart.eps}}

  \caption{Analyzing pipeline illustrating the use of RNAz and the helper
    programs.  (1) \texttt{rnazWindow.pl} slices the input alignments in
    overlapping windows and performs a variety of filtering and
    pre-processing steps. (2) The processed alignments can be scored with the
    \texttt{RNAz} program (3) Overlapping hits are merged with
    \texttt{rnazCluster.pl}. In addition, all relevant data is extracted from
    the raw output and stored in a tabulator delimited data file. Using the
    \texttt{--html} option, \texttt{rnazCluster.pl} generates a tree of HTML
    pages with illustrations of the predicted structures. You need additional
    software for this step to work. (4) The results can be filtered, sorted
    and annotated in various ways. All programs read a tab-delimited file and
    write a tab-delimited file. (5) Using \texttt{rnazIndex.pl}, the
    tab-delimited data files can be exported to standard formats as GFF and
    BED. It is also possible to create a HTML formatted index file for the
    optional HTML output created in step 3.}
\label{fig:pipeline}
\end{figure}


\subsubsection{Choosing raw input alignments}

Choosing a reasonable set of input alignments is one of the most important
steps during the analysis. There are a variety of different programs
available to generate genome-wide alignments. Here, we use Multiz
alignments of up to seven \emph{Saccharomyces} species which can be
downloaded from the UCSC genome browser (\url{genome.ucsc.edu}). In
principle, we could use \emph{all} alignments covering the complete genome.
The biggest problem in large genomic screens is probably specificity. We
have a relatively constant background signal of false positives. The more
sequences we put into the screen, the more false positives we get out. It
is, therefore, a good idea to choose the input set as small as possible
(trying not to discard any interesting regions of course). In our case, we
only analyze the intergenic regions, i.e. we discard any coding regions and
all other annotated features (pseudogenes, repeats, ARS elements, \dots).
We retain known ncRNAs as positive control in the set.  The selection was
easily accomplished using the ``Table browser'' feature of the genome
browser. We finally obtained a MAF alignment (\texttt{input.maf}) with
10,822 alignment blocks, covering 983,947 bases of the genome (see
section~\ref{sec:estim-false-posit-1} how to get these numbers out of a MAF
file.).

\subsubsection{Pre-processing raw alignments}

As described in section~\ref{sec:scor-long-alignm}, it is necessary to
score long alignments in overlapping windows. Given the partly poor quality
of automatically generated genome-wide alignments additional pre-processing
steps are required to filter out gap-rich regions, dubious aligned
fragments or low complexity regions. All pre-processing is done by the
\texttt{rnazWindow.pl} program which, per default, performs the following
steps:

\begin{enumerate}

\item Slice alignments in overlapping windows of size 120 and slide 40.
\item Check each pairwise alignment of the reference sequence (= first
  sequence) to all other sequences and, after removing common gaps, discard
  sequences with more than 25\% gaps in this pairwise alignment.
\item Discard any sequences which are outside the definition range of
  RNAz (e.g. $<$50 nucleotides, GC content $>$0.75). 
\item Discard the complete alignment if either the reference sequence was
  discarded in a previous step or only the reference sequence is left (i.e.
  number of sequences $<$2)
\item If the number of sequences is $>$6, choose a subset of 6 sequences
  with mean pairwise identity optimized to a target value of 80\%.
\item Remove all sequences which are 100\% identical. Never remove the
  reference sequence and if all sequences are identical retain only a
  pairwise alignment.

\end{enumerate}

All these steps can be customized with the appropriate command-line
parameters. Here we use the default settings. We define, however, a minimum
number of four sequences in the alignment retaining only regions which are
well conserved across several species:

\begin{verbatim}
# rnazWindow.pl --min-seqs=4 input.maf > windows.maf
\end{verbatim}

This command will take a few minutes.

\subsubsection{Running RNAz}

The file \texttt{windows.maf} is now ready for being scored with RNAz. We
use the \texttt{--both-strands} parameter to score both the forward and the
reverse complement strand. We also set \texttt{--show-gaps} which means
that the output is shown including the gaps. With this option it is
possible to recover the complete alignment from the RNAz output file which
is useful in later steps of the pipeline. Finally, we set a $P$ value
cutoff of 0.5, meaning that only positive predictions are stored resulting
in a much smaller output file.

\begin{verbatim}
# RNAz --both-strands --show-gaps --cutoff=0.5 windows.maf > rnaz.out
\end{verbatim}

This will take approximately one hour on a modern desktop computer but may
vary depending on your system.

\subsubsection{Clustering the results}
\label{sec:clustering-results}

The file \texttt{rnaz.out} now holds all windows that have a positive RNAz
signal with $P>$0.5. It is possible that several windows cover the same
genomic region. Overlapping windows are therefore clustered in \emph{loci}:

\begin{verbatim}
# rnazCluster.pl rnaz.out > results.dat 
\end{verbatim}

This command assigns each window a consecutively numbered ``window ID'' and
each group of overlapping windows a ``locus ID''. For each window and each
locus all relevant data (use \texttt{--help} for details) is stored in a
tabulator separated text file. 

Inspecting the file \texttt{results.dat}, we see that we have 1104 windows
which can be grouped in 454 loci.

It is important to note that the term ``locus'' must \emph{not} be
understood in the sense of a genetic unit. It is, of course, possible that
several loci of our procedure cover one long ncRNA gene.

At this point we also want to add that we are painfully aware of the fact
that the process of first slicing the alignments and the re-cluster them is
not optimal. Ideally one would like to predict conserved RNA structures
\emph{locally} without sliding windows. Although this should be possible
\cite{hofacker04} and we are working on a local version of RNAz, the
sliding window approach is currently the only reasonable protocol.

\subsubsection{Filtering and sorting the results}
\label{sec:filt-sort-results}

The data file now contains the raw data of all hits. In the following
analysis steps, one usually wants to filter and sort candidates by various
criteria. For this purpose you can use the programs \texttt{rnazFilter.pl}
and \texttt{rnazSort.pl}. For example, 

\begin{verbatim}
# rnazFilter.pl "P>0.9" results.dat
\end{verbatim}

lists all windows that have a $P$-value higher than 0.9.  For hints on how
to formulate more complex filtering expressions see
Note~\ref{sec:formulating-filter}. With the \texttt{--count} option you can
count the hits. We have 670 Windows in 303 loci on the $P$>0.9 significance
level. In addition, we can sort the hits:

\begin{verbatim}
# rnazFilter.pl "P>0.9" results.dat | rnazSort.pl combPerPair
\end{verbatim}

This sorts the output by the ``Combinations/Pair'' value, i.e. by
compensatory mutations supporting the structure (explained in
Note~\ref{sec:addit-outp-valu}).

\subsubsection{Exporting the results to standard annotation formats}
\label{sec:export-results-stand}

Using different combinations of \texttt{rnazFilter.pl} and
\texttt{rnazSort.pl} you can create various sub-selections of the complete
data from \texttt{results.dat}. You always get a tabulator delimited
data-file. The program \texttt{rnazIndex.pl} helps you to convert these
kind of data files into the standard annotation formats GFF
(\texttt{--gff}) or BED (\texttt{--bed}). GFF
(\url{http://www.sanger.ac.uk/Software/formats/GFF/}) is a widely used
format supported by many programs. BED
(\url{http://genome.ucsc.edu/FAQ/FAQformat}) is the native annotation
format for the UCSC genome browser but is generally useful because of its
simplicity (in its simplest form it is a list of genomic locations:
\texttt{sequenceID start stop}).

The following command creates a GFF file from all results:

\begin{verbatim}
# rnazIndex.pl --gff results.dat > results.gff
\end{verbatim}

\subsubsection{Visualizing the results on a website}
\label{sec:website}

It is often insightful to manually check individual predictions, for
example by analyzing different illustrations of consensus structures (see
Note~\ref{sec:filt-sign-hits}). The creation of the necessary files is a
tedious task which, however, can easily be automatized. If you run the
cluster command from section~\ref{sec:clustering-results} with the option
\texttt{--html},

\begin{verbatim}
# rnazCluster.pl --html rnaz.out > results.dat 
\end{verbatim}

the program generates image files for all hits. For the \texttt{--html}
option to work, you need to have installed the Vienna RNA package
(including the Perl programs of the \texttt{Utils} directory) and the
program Ghostscript, see section \ref{sec:optional-software}.
\texttt{rnazCluster.pl} creates a subdirectory called \texttt{results},
which, in turn, has a subdirectory \texttt{locusN} for each locus. In the
\texttt{locusN} directories you find the image files together with an
\texttt{index.html} which arranges the images for each locus on a web-page.
You can open the index files using your favorite web-browser.

To get an HTML formatted table of all hits linking to the sub-pages for
each locus, you can use \texttt{rnazIndex.pl} with the \texttt{--html}
option:

\begin{verbatim}
# rnazIndex.pl --html results.dat > results/results.html
\end{verbatim}

\subsubsection{Comparing hits to known annotation}

Once you have a list of predicted RNAs, you may want to add additional
annotations to your predictions. You can simply add additional fields to
the tabulator separated data file at your convenience. Here we demonstrate
this by comparing our prediction with the known ncRNA annotation from the
\emph{Saccharomyces} genome database. The program \texttt{rnazAnnotate.pl}
checks each predicted locus for overlap with an annotation file in BED
format:

\begin{verbatim}
# rnazAnnotate.pl --bed ../sgdRNA.bed results.dat > annotated.dat
\end{verbatim}

We find that out of 454 predicted loci, 280 overlap with known ncRNAs (of
the 303 loci with $P>0.9$, 215 are known ncRNAs). We detect all sorts of
different ncRNA classes (tRNAs, rRNA, snRNAs, snoRNAs, RUFs
\cite{mccutcheon03}, and other ncRNAs like telomerase RNA or RNAseP,\dots)
Most of the known 373 ncRNAs in yeast are tRNAs (275), which are partly
difficult to detect in this screen because most of them are $\approx$ 100\%
conserved (i.e. no covariance information).

Without providing a detailed sensitivity analysis for this specific yeast
screen, we want to add that sensitivity highly depends on the ncRNA class.
MicroRNAs, for example are easy to detect because of the high thermodynamic
stability of the hairpin precursor. On the other hand, C/D type snoRNAs for
example are generally difficult to detect because they lack a pronounced
secondary structure. We miss completely ncRNAs which do not depend on a
secondary structure for their function, as for example the yeast
\emph{SER3} regulating RNA \cite{martens04} which, as expected, does not
show up in this screen.


\subsubsection{Annotating hits with database search}
\label{sec:blast}

Another possibility to annotate predicted ncRNAs is to compare the
sequences to databases of known ncRNAs. In the following we match the
predicted loci against the Rfam database \cite{griffiths-jones05} using a
simple Blast sequence search. Alternatively, one could use more sensitive
methods which also incorporate secondary structural information (e.g.
Infernal \cite{eddy02}).  To run this example, you need the
\emph{S.cerevisiae} sequence files, the Rfam database file and a working
NCBI Blast installation.

First change into the directory \texttt{rfam} and run:

\begin{verbatim}
# formatdb -t rfam -i rfam -p F
\end{verbatim}

This command creates the index files for the file \texttt{rfam}, which is a
Fasta formatted file with all entries of the database. You now can run:

\begin{verbatim}
# rnazBlast.pl --database rfam --seq-dir=seq \
               --blast-dir=rfam results.dat >annotated.dat
\end{verbatim}

This program takes the \emph{S. cerevisiae} reference sequence for each
locus and runs a Blast search against the Rfam database.  If there is a hit
with an expectation value below some cutoff (default: $E<10^{-6}$), the
name of the matching database query is added as a new field to the data
file. Please note that you have to specify the locations of the sequence
data files and the blast index files on the command line.


\subsubsection{Estimating false positives and gathering statistics}
\label{sec:estim-false-posit-1}

To get an impression of the false-positive rate of a specific screen it is
useful to do a control screen on randomized alignments. The command

\begin{verbatim}
# rnazRandomizeAln.pl input.maf > random-input.maf
\end{verbatim}

will produce a randomized version of the input alignments by shuffling the
positions in the alignments. The program aims to remove any correlations
arising from a natural secondary structure while preserving important
alignment and sequence characteristics as for example mean pairwise
identity or base composition \cite{washietl04}.

We repeated the complete analysis with the randomized alignments and we get
102 and 39 loci, on the $P>$0.5 and $P>$0.9 level, respectively. 

\begin{table}
  \caption{Statistics of the yeast example screen}

  \begin{center}
    \begin{tabular}{lcc}\toprule
      & $P$>0.5 & $P$>0.9\\
      \midrule
    Predicted loci & 454 & 303\\
    Known ncRNAs & 280 & 215\\
    Loci without annotation & 174 & 88\\
    Predicted bases & 60,834 & 44,082\\
    Fraction of input alignments (\%) & 10.6 & 7.7\\
    Predicted loci random & 102 & 39\\
    Predicted bases random& 12,823 & 6,017\\
    Fraction of input alignments random (\%) & 2.2 & 1.0\\
    \bottomrule
  \end{tabular}
\end{center}
\label{tab:statistics}
\end{table}


Table~\ref{tab:statistics} summarizes all results of this example screen.
There are a few programs which help you to gather statistics on your data.
For example,

\begin{verbatim}
# rnazIndex.pl --bed results.dat \
               | rnazBEDsort.pl | rnazBEDstats.pl
\end{verbatim}

gives you detailed information on the predicted loci, including the covered
genomic region in nucleotides. This command first exports the results as
BED file, sorts the results by the genomic location and, finally, evaluates
the coordinates in the BED file. If you want to get statistics on your
input alignments, you can use a command like this:

\begin{verbatim}

# rnazMAF2BED.pl --seq-id=sacCer windows.maf \
                 | rnazBEDsort.pl | rnazBEDstats.pl

\end{verbatim}

\texttt{rnazMAF2BED.pl} converts a MAF formatted alignment file to
coordinates in BED format. With \texttt{--seq-id} you specify which
sequence is used as reference.

Using these tools, you find for example that in the random control 1.0\% of
the input sequences are predicted as RNA on the $P>$0.9 level. This is
exactly the false positive rate as expected (section
\ref{sec:understanding-output}). The absolute number of false positives,
however, strongly depends on your specific screen. In this example we have
88 hits $P>$0.9 without RNA annotation and find that 39 hits should be
expected by chance. So we must expect that roughly half of our predictions
are false positives. On the other hand, this implies that the other half of
the predicted loci should be real functional RNA structures, either as part
of a ncRNA or as regulatory element of a mRNA. However, one always have to
bear in mind possible shortcomings of this kind of random control, see
Note~\ref{sec:estim-false-posit}.



\section{Notes}

\subsection{Custom installation of RNAz}
\label{sec:cust-inst-rnaz}

The installation process using \texttt{./configure} and \texttt{make}
should work on all UNIX-like systems. If you get error messages it may be
necessary that you install additional ``developer packages''. On some Linux
distributions, for example, there is no C-compiler installed by default.
Also on OS X it is necessary that you have installed the ``XCode'' tools.

If you do not have root privileges or want to install RNAz into a different
location than \texttt{/usr/local/} (e.g. your home directory) you can use
the following command:

\begin{verbatim}
# ./configure --prefix=/home/stefan --datadir=/home/stefan/share
\end{verbatim}

This installs the executable to \texttt{/home/stefan/bin} and the example
files, Perl programs and other data to \texttt{/home/stefan/share/RNAz}.
Please note that the \texttt{bin} directory must be in your \texttt{PATH}
of executables if you want to call the \texttt{RNAz} executable without
specifying the complete path.

\subsection{Running the Perl programs}
\label{sec:inst-runn-perl}

Since different people usually like to have their scripts in different
locations, the Perl programs are \emph{not} installed to
\texttt{/usr/local/bin} by default. They are installed to\\
\texttt{/usr/local/share/RNAz/perl}. To make them available from other
locations, copy all files from this directory to a directory which is
included in your \texttt{PATH} of executables, e.g.:

\begin{verbatim}
# cp /usr/local/share/RNAz/perl/* /usr/local/bin
\end{verbatim}

Alternatively, you can add the directory with the Perl programs to your
\texttt{PATH} variable by editing your \texttt{.bashrc} or \texttt{.cshrc}
file in your home directory.

In any case, it is important that the Perl module file \texttt{RNAz.pm}
resides in the same directory as the Perl programs (\texttt{*.pl}). All the
Perl programs depend on this module file.

Another important point is, that the Perl programs expect that the path of
the Perl executable is \texttt{/usr/bin/perl}. This is the standard
location on almost all Linux/UNIX systems and OS X. If your Perl
installation is different you have to customize the first line of all the
Perl programs according to the location of your \texttt{perl} executable.

On a Windows system the Perl programs should work if you have installed
Perl as described in section~\ref{sec:perl} and set the \texttt{Path}
variable as described in section~\ref{sec:microsoft-windows}.

\subsection{Optional software on Windows on OS X}
\label{sec:opti-softw-wind}

To install the necessary software for the programs in sections
\ref{sec:website} and \ref{sec:blast} might be a bit tricky on Windows and
OS X.

You can install the Vienna RNA package and NCBI Blast without problems on
OS X by following the instructions in section \ref{sec:inst-opti-softw}.
However, unlike on a Linux system, Ghostscript is not installed per
default. You can try to get a pre-compiled package from
\url{fink.sourceforge.net} or \url{darwinports.opendarwin.org}.
Alternatively, you can download the source from
\url{http://www.ghostscript.com/} and build the package with
\texttt{./configure} and \texttt{make}.

On Windows you can install Ghostscript through a simple installer file
which you can download from \url{http://www.ghostscript.com/}. Follow the
installation instructions. Locate the newly installed file
\texttt{gswin32c.exe} and copy it to a folder which is in your
\texttt{Path} (e.g.  the folder, where the \texttt{RNAz.exe} executable
resides).  Rename the file to \texttt{gs.exe}.

Windows users do not have to install the Vienna RNA package. The relevant
programs are part of the RNAz windows installer.

To install NCBI blast on windows, create a new folder (e.g.
\texttt{c:$\backslash$Program Files$\backslash$blast}) and download the
\texttt{blast-2.*-win32.exe} file from \url{ftp://ftp.ncbi.nih.gov/blast}.
Within the new folder, double click on the \texttt{blast-2.*-win32.exe}
file which extracts the programs and data. Add the \texttt{bin}
subdirectory to your \texttt{Path}: Right-click \emph{My Computer}, then
click \emph{Properties}.  Select \emph{Advanced}/\emph{Environment
  variables}/\emph{New}. Add the complete path of the blast \texttt{bin}
directory to the variable \emph{Path}, use ``\texttt{;}'' as separator.


\subsection{Creating the input alignments}
\label{sec:creat-input-alignm}

RNAz can only detect a conserved structure if this structure is accurately
reflected in the alignment. Therefore, the quality of the alignment is
crucial for the success of the analysis. In practice, we found that if your
alignment has a mean pairwise identity above appr. 60\% simple sequence
based progressive, global alignment methods yield reasonable results and
there is not much difference between methods. One of the best programs for
aligning RNAs is Clustal W. For genome-wide alignments we have only
experience with Multiz alignments. Also these alignments are of reasonable
quality and there is generally no need for re-alignment. We suppose that
also other genome-wide alignment methods produce suitable alignments as
long the aligned regions are of sufficient similarity (mean pairwise
identity somewhere around $60\%$ or above). In cases with sequences below
60\% identity, simple sequence based methods usually do not find an optimal
\emph{structural} alignment. Altough in principle structural enhanced
alignments could help here, this alternative is not relevant in practice.
First, there are hardly any structural multiple sequence alignment programs
available. Second, current approaches are much too slow to use them for
every-day analysis. Third, RNAz is not trained on structural alignments. In
contrast to pure sequence based alignment, you would get unusual high SCIs.
This could confuse the decision model and you would get unpredictable
results.

\subsection{Additional output values}
\label{sec:addit-outp-valu}

The consensus MFE which is calculated by the RNAalifold algorithm (see
section~\ref{sec:struct-cons}) can be split in two terms.  One is the
``energy contribution'', which is the folding energy from the standard
energy model. The ``covariance contribution'' is the part which comes from
the additional ``bonus'' or ``penalty'' energies for
compensatory/consistent and inconsistent mutations, respectively. If the
covariance term is negative, there are more compensatory mutations than
inconsistent mutations.

RNAz also calculates another value quantifying compensatory/consistent
mutations: ``Combinations/Pair''. This is the number of \emph{different}
base pair combinations in the consensus structure divided by the number of
pairs in the consensus structure. Both the covariance contribution of the
consensus MFE and the ``Combinations/Pair'' are mainly useful for final
sorting a set of equally good predictions with have been filtered using
other criteria (e.g. $P$ or $z$-scores).

RNAz uses a SVM algorithm for classification.  The raw output of the SVM is
the so-called ``decision-value''. This real-valued number is positive if
the prediction is ``RNA'' and negative otherwise.  From this value we
calculate the more intuitive ``RNA class probability'' or ``$P$-value''
which is 0.5 for a decision value of 0. In some cases, the raw decision
value can be more convenient than the $P$ value (e.g. if you want to plot
the distribution of RNAz results).

\subsection{Estimating false positives}
\label{sec:estim-false-posit}

The RNAz classification model is trained on a test set consisting of
natural RNAs as positive examples and randomly shuffled alignments as
negative examples. Thus, any signal reported by RNAz is relative to an
\emph{artificial} background. Although this null model of shuffled
sequences is probably the most sensible choice possible, one cannot assume
that it behaves exactly like the \emph{natural} background of real sequence
data. Also the estimation of false positive rates is based on shuffled
sequences. We want to stress that, therefore, such an estimation of false
positives must be regarded as a lower bond since one cannot rule out the
possibility that non-random patterns in natural sequences cause a higher
rate of false positives than one observes in synthetic random sequences. In
particular, the $z$-score calculation might be affected by such effects.
For example di-nucleotide content could bias the MFE structure prediction.
As an opposite effect one must consider the possibility that the shuffling
procedure cannot remove all secondary structure signals and thus
\emph{over}estimates the real false positive rate. If you shuffle an
alignment with many compensatory mutations, the number of ``compatible
columns'' stays the same, allowing for compensatory mutations also in the
shuffled alignment.

\subsection{Manual inspection of candidates}
\label{sec:filt-sign-hits}

If you have a hit with $P>$0.9, you have approximately a chance of 1 in
100, that this arises through pure chance (but see als
Note~\ref{sec:estim-false-posit}). It makes sense to critically look at a
hit. Sometimes the signal only comes from a low $z$-score of borderline
significance and there is no evidence for structural conservation.
Sometimes the complete alignment looks pathological (weird gap patterns,
low complexity regions etc.) which suggests that this is not a relevant
structure. It is useful to analyze a predicted structure with RNAalifold
and its visualization methods. Visual inspection of a color coded alignment
and the consensus structure gives you an idea about compensatory mutations
supporting the structure and inconsistent mutations which do not support
the structure. It must be noted that many ncRNAs in real life-data are not
supported by compensatory mutations, still they can be detected based on
the stability and/or the SCI. The SCI implicitely also considers the
mutational pattern outside of stems. To conclude, the $P$ value efficiently
filters your data for candidates, but only the complete picture can help
you in your decision on the relevance of a hit.

\subsection{Advanced filtering}
\label{sec:formulating-filter}

Filtering the tab-delimited data files using standard UNIX tools like
\texttt{grep} or \texttt{awk} is difficult because of the special
window/locus grouping of the data. You can use the \texttt{rnazFilter.pl}
program. The filter statement uses the field names (e.g.
\texttt{z,SCI,combPerPair}, see \texttt{--help} for a complete list) and
standard logical operators as used in the Perl language: \texttt{>}
(greater than), \texttt{<} (smaller than),\texttt{==} (equals numerically),
\texttt{eq} (equals string), \texttt{not}, \texttt{and}, \texttt{or},
\texttt{=\raisebox{-0.7ex}{\~{ }}/regex/} (pattern match).  In addition you
can use brackets to group and combine statements. For example the following
statement gives you all windows with $P>0.9$ and $z<-3$ on chromosome 13:

\begin{verbatim}
# rnazFilter.pl "P>0.9 and z<-3 and seqID=~/chr13/" results.dat
\end{verbatim}

It is important that \emph{everything} you put in the filter statement is
evaluated by the Perl interpreter. This can be potentially harmful, so take
care.

\clearpage

\bibliographystyle{custom}
\sf \small
\bibliography{manual}
\normalfont
\normalsize

\appendix

\section{Manual pages}
\label{sec:manual-pages}

\input{man.tex}






\end{document}
